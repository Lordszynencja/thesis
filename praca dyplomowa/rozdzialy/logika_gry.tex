\chapter{Logika gry}
\thispagestyle{chapterBeginStyle}
\label{chapter_logic}

W tym rozdziale omówię w jaki sposób działa silnik gry i mechanizmy umożliwiające poprawne działanie wszystkich jej aspektów.

Głównym obiektem w grze jest obiekt klasy Game, inicjalizujący wszystkie inne obiekty oraz synchronizujący ich działanie.

Inicjalizacja zaczyna się od utworzenia obiektów odpowiedzialnych za poszczególne elementy gry, czyli logiki, grafiki, muzyki i sterowania, opisanych w dalszej czę\'sci pracy.

\begin{wrapfigure}{l}{0.39\textwidth}
	\centering
	\noindent\includegraphics[width=0.39\textwidth]{game_init_uml}
	\caption{diagram przedstawiający interakcję gracza z grą}
\end{wrapfigure}
Następny krok to wczytanie ustawień, takich jak gło\'sno\'sć i ustawienia graficzne, a także stanu gry - odblokowanych poziomów, zdobytych punktów, uzyskanych osiągnięć i kilka innych statystyk. Ustawienia te są wczytywane z udostępnianego przez przeglądarkę magazynu lokalnego, dostępnego w języku HTML5. Magazyn lokalny jest podobny do używanych na większo\'sci stron ciasteczek, ale jest większy i łatwiejszy w użyciu, ponieważ jest obiektem działającym jak słownik z języka JavaScript, zawierającym tylko klucze i ciągi znaków do nich przypisane. Dzięki temu można łatwo zapisać w nim obiekt z danymi po przetłumaczeniu go na format JSON. Przy wczytaniu danej typu string z magazynu lokalnego, wystarczy zamienić dane w formie JSON-a na zwykły obiekt JavaScript.

Dane przechowywane w magazynie lokalnym sa przypisane do strony i są w nim przechowywane dopóki ich nie usuniemy lub użytkownik nie wyczy\'sci magazynu, inaczej niż w ciasteczkach, gdzie trzeba było podać datę przydatno\'sci, po której dany obiekt był usuwany. Ciasteczka mogły też pomie\'scić dużo mniej danych.

Istnieje odmiana magazynu lokalnego, magazyn sesji, w którym dane są obecne dopóki nie zamkniemy danej zakładki.

W przypadku braku danych w magazynie lokalnym używane są dane domy\'slne.

Po wczytaniu danych, następuje rozpoczęcie gry. Ustawiany jest timer, co 20 ms, czyli 50 razy na sekundę, uruchamiający funkcję obliczającą jeden cykl silnika gry. Następnie do przeglądarki jest wysyłane żądanie uruchomienia funkcji renderującej. Argument funkcji requestAnimationFrame jest uruchamiany wtedy przed następnym od\'swieżeniem rysowanych elementów (takich jak element Canvas, na którym jest umieszczany obraz z gry). Na koniec uruchamiana jest muzyka z głównego menu gry.

Po skończeniu inicjalizacji gry, okresowo wywołują się funkcje odpowiedzialne za logikę gry, opisaną poniżej, oraz za grafikę, opisaną w rozdziale \ref{chapter_graphics}.
\newpage

\noindent{\LARGE Menu gry}

\begin{wrapfigure}{r}{0.39\textwidth}
	\centering
	\noindent\includegraphics[width=0.39\textwidth]{menus_uml}
	\caption{Przej\'scia między menu}
\end{wrapfigure}
Pierwsze co gracz widzi po włączeniu gry, to ekran głównego menu, zawierający opcje ''Graj'' oraz ''Opcje''. Sterowanie w menu to klawisze strzałek, enter i escape, służące do zmiany pod\'swietlonej opcji, wej\'scia do daneg menu oraz cofnięcia się z niego. Wybranie pozycji ''Opcje'' przeniesie nas do menu z różnymi ustawieniami gry, takimi jak ilo\'sć efektów, gło\'sno\'sć czy też możliwo\'sć zresetowania postępu w grze. Druga opcja natomiast przenosi nas do menu wyboru poziomu, będzie tam też można za zdobyte w grze punkty ulepszyć statek gracza. Na początku odblokowany jest tylko jeden poziom, ale z postępami w grze gracz odblokowuje kolejne poziomy, a także możliwo\'sci rozwoju statku. Po przej\'sciu do którego\'s z poziomów zaczyna się wła\'sciwa gra, na dole ekranu pojawia się statek gracza, a z góry zaczynają po chwili nadlatywać przeciwnicy. Wci\'snięcie spacji powoduje strzelanie z głównych działek, a gdy trafimy w przeciwnika wystarczająco dużo razy, zastaje on pokonany a gracz dostaje punkty. Aby przej\'sć poziom wystarczy dolecieć do jego końca, omijając lub pokonując spotkanych po drodze przeciwników. Aby zatrzymać grę należy wcisnąć przycisk escape. Menu pauzy pozwala przej\'sć do panelu opcji, wrócić do gry lub skończyć poziom.\bigskip

\noindent{\LARGE Główny silnik gry}\smallskip

Główny silnik gry działa poprzez aktualizowanie stanu gry co każde 20 milisekund. Jeden taki cykl polega zazwyczaj na uaktualnieniu wszystkich aktywnych obiektów. Gdy gracz jest w głównym menu gry, nie ma zbyt wiele obiektów do aktualizacji, dopiero gdy włączy się który\'s z poziomów pojawiają się większe, bardziej skomplikowane obiekty. Na początku jest aktualizowany obiekt klasy UI, który zajmuje się obsługiwaniem aktualnego menu (poziom w grze to także rodzaj menu). Następny jest obiekt osługujący grafikę (tutaj aktualizacja polega zazwyczaj na wyczyszczeniu buforów), a potem wszystkie nowe dane do rysowania są przekazywane do obiektu obsługującego grafikę. Na końcu raz na dwie sekundy aktualne statystyki i opcje gry są zapisywane w magazynie oraz zostaje zwiększony licznik czasu.

Większo\'sć klas w projekcie ma funkcje ''update'' oraz ''draw'', służące odpowiednio do aktualizacji oraz rysowania obiektu.

Niezależnie od aktualizacji działa sterowanie w grze. Jest ono przekazywane na dwa sposoby: jako tablica wci\'sniętych klawiszy przechowywana w odpowiednim obiekcie oraz zdarzenie wysyłane do aktualnego menu. To drugie służy tylko do sterowania w menu, i nie jest używane we wła\'sciwej grze poza opcją włączenia menu pauzy. Tablica klawiszy służy natomiast podczas aktualizacji w samej grze, i zawiera informację typu boolean dla każdego klawisza, mówiącą czy jest on wci\'snięty.\bigskip

\noindent{\LARGE Aktualizacja w trakcie poziomu}

Aktualizacja w trakcie poziomu zaczyna się od aktualizacji obiektu danego poziomu, zazwyczaj powoduje to pojawienie się przeciwników oraz kończenie misji przy spełnieniu odpowiedniego warunku. Każdy poziom ma własna klasę reprezentującą go.\\
Każda klasa poziomu zawiera przypisane do niego: \begin{itemize}[topsep=0.2em, itemsep=0.5em, partopsep=0em, parsep=0em]
	\item listę zawierającą dane o typie, czasie pojawienia się i zachowaniu przeciwników
	\item długo\'sć poziomu (gdy poziom się kończy, gracz wygrywa)
	\item informacje o specjalnym przeciwniku na końcu poziomu (lub jego braku)
\end{itemize}\smallskip

\noindent Następny krok to aktualizacja wszystkich efektów, najczę\'sciej przez zmianę ich pozycji. Efekty są opisane dokładniej w rozdziale \ref{chapter_graphics}.\newpage
\noindent Następnie aktualizują się przeciwnicy, czyli:\begin{itemize}[topsep=0.2em, itemsep=0.5em, partopsep=0em, parsep=0em]
	\item zmieniają swoją pozycję
	\item używają danych im broni
	\item aktywują umiejętno\'sci specjalne
\end{itemize}
Każdy przeciwnik ma przypisane do siebie swoją pozycję, zachowanie, model statku oraz zazwyczaj broń lub bronie. Każda klasa typu przeciwnik ma także punkty uzyskane za pokonanie danego typu przeciwnika, jego maksymalną wytrzymało\'sć  oraz domy\'slny model używany przy wykrywaniu kolizji.\smallskip

Kolejny krok aktualizuje pociski przeciwników, czyli zmienia ich położenie według zaprogramowanego ruchu, a także sprawdza czy nie zderzyły sie one z graczem. System kolizji jest opisany w rozdziale \ref{chapter_collission_detection}. Potem to samo jest powtarzane dla pocisków wystrzelonych przez gracza.

Obiekty pocisków zawierają zazwyczaj informacje o położeniu, obrażeniach, sposobie ruchu i modelu używanym do zderzeń.\smallskip

Ostatnim krokiem jest aktualizacja statku gracza. Sprawdzane jest tam, czy gracz wcisnął klawisze służące do przemieszczania statku (strzałki). Na końcu aktualizowany jest statek gracza, w tym to czy gracz używał broni (klawisz spacji), czy chłodnica statku się nie przegrzała oraz czy statkowi nie skończyły sie punkty wytrzymało\'sci, co wcze\'snie kończy grę.
Obiekt typu Player jest jednym z bardziej skomplikowanych. Oprócz zwykłych danych takich jak wygląd, pozycja oraz model, zawiera także dokładniejsze informacje o broni. W przeciwieństwie do przeciwników jest też możliwo\'sć zmienić jego statek i broń, a na jego zachowanie oprócz samego sposobu poruszania się ma wpływ także sterowanie gracza. Ze względu na to że gracz widzi go cały czas jest jedną z najważniejszych czę\'sci gry.\bigskip

Gdy wszystkie kroki zostały wykonane, jeden cykl silnika gry się kończy i gra czeka na następny. W międzyczasie między aktualizacjami obraz jest także przerysowywany, co jest opisane w rozdziale \ref{chapter_graphics}.\newpage

\begin{wrapfigure}{r}{\textwidth}
	\centering
	\noindent\includegraphics[width=\textwidth]{UI_uml}
	\caption{przykład obiektu UI, pokazujący strukturę obiektów}
\end{wrapfigure}

\cleardoublepage