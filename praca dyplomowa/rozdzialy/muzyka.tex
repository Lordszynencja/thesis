\chapter{Muzyka i efekty dźwiękowe}
\thispagestyle{chapterBeginStyle}
\label{ch:music}

Ten rozdział opisuje jak działa muzyka i efekty dźwiękowe w mojej grze.\vspace{10pt}\\
\noindent{\Large Dźwięki}\smallskip

Wszystkie efekty dźwiękowe oraz muzyka przechowywane są w plikach skompresowanych za pomocą MPEG Audio Layer 3 czyli popularnym formacie mp3. Każdy dźwięk, który ma zostać odtworzony, musi najpierw być pobrany z serwera.\\

Do odtwarzania plików dźwiękowych używam klasy Audio, dostępnej w piątej wersji języka HTML. Przed tą wersją odtwarzanie dźwięku wymagało użycia wtyczki, ale dzięki rozwojowi języka HTML jest to teraz dostępne jako czę\'sć standardu.\\

\noindent{\Large Efekty dźwiękowe}\smallskip

Wszystkie dźwięki w mojej grze są obsługiwane przez napisaną przeze mnie klasę Sound. Wszystkie nazwy dźwięków są przechowywane w słowniku w tej klasie. Funkcja służąca do odtwarzania efektów dźwiękowych wygląda tak:
\begin{lstlisting}
play(name, volume) {
	this.activeSound++;
	var a = this.activeSound;
	this.playing[a] = new Audio("sounds/" + this.sounds[name]);
	this.playing[a].onended = this.defaultOnEnded(a);
	this.playing[a].volume = volume*conf.overallVolume*conf.effectsVolume;
	this.playing[a].play();
}
\end{lstlisting}
Jako argumenty funkcja przyjmuje nazwę efektu dźwiękowego i jego gło\'sno\'sć.

Aby można było odtwarzać wiele dźwięków naraz, wszystkie dźwięki są umieszczane w tablicy. Dzięki użyciu tablicy wszystkich odtwarzanych dźwięków, można mieć dostęp i wyłączyć każdy z nich (na przykład przy resetowaniu poziomu, wracaniu do menu itp.). Aby nie było kolizji, przy każdym nowym dźwięku zwiększany jest licznik dźwięków.

Potem tworzony jest nowy dźwięk z pliku, którego nazwa jest warto\'scią w słowniku dla klucza w zmiennej name. Funkcją włączana po zakończeniu danego dźwięku usuwa go. Końcowa gło\'sno\'sć dźwięku jest iloczynem podanej gło\'sno\'sci, całkowitej gło\'sno\'sci gry oraz gło\'sno\'sci efektów dźwiękowych. Na koniec dźwięk jest odtwarzany.\newpage

\noindent{\Large Muzyka}\smallskip

Muzyka zachowuje się nieco inaczej niż efekty dźwiękowe. Przede wszystkim, w dowolnym momencie jest odtwarzana tylko jedna \'scieżka dźwiękowa dla muzyki, jest za to ona często zamieniana na inną, na przykład przy przej\'sciu z głównego menu do którego\'s poziomu.\\
Funkcja zmieniająca odtwarzaną muzykę prezentuje się następująco:
\begin{lstlisting}
changeMusic(music, fadetime = 1) {
	var thisvar = this;
	if (this.changingMusic) {
		return;
	} else if (this.music == null) {
		this.changingMusic = true;
		this.playMusic(music);
	} else {
		this.changingMusic = true;
		var musicVolume = this.musicVolume;
		var fadetimeScaled = fadetime*100;
		for (var i=0;i<fadetimeScaled;i++) {
			window.setTimeout((function(volume) {
				return function() {
					thisvar.setMusicVolume(volume);
				}
			})((1-i/fadetimeScaled)*musicVolume), 10*i);
		}
		window.setTimeout(function() {
			thisvar.playMusic(music);
			thisvar.setMusicVolume(1);
		}, fadetimeScaled*10);
	}
}
\end{lstlisting}
Dzięki zastosowaniu zmiennej typu boolean, podczas zmiany muzyki nie można przypadkowo zacząć zmieniać muzyki na jeszcze inną, co mogłoby powodować odtworzenie dwóch piosenek naraz.\\
Dodatkowo muzyka jest stopniowo wyciszana żeby uzyskać efekt Fade, po czym odtwarzany jest nowy utwór. Czas trwania efektu Fade jest modyfikowalny.
\cleardoublepage