\chapter{Analiza zagadnienia}
\label{chapter_analysis}
\thispagestyle{chapterBeginStyle}

W tym rozdziale przedstawię założenia i oczekiwane efekty dla różnych aspektów gry.

Przede wszystkim, gra ma być skonstruowana tak aby działała na wszystkich nowszych i najczę\'sciej używanych przeglądarkach, takich jak Google Chrome, Mozilla Firefox, Internet Explorer i Opera.
Gra powinna także odpowiednio wyglądać i zachowywać się zgodnie z tym czego można by sie po niej spodziewać, na przykład je\'sli dwa obiekty wyglądają jakby powinny się zderzyć, to rzeczywi\'scie powinno zaj\'sć zderzenie.

Główny element gry to silnik gry. Zajmuje się on  wykonywaniem odpowiednich funkcji na obiektach w grze.
Zazwyczaj jest to pętla działająca tak długo jak działa gra, używająca timer-ów do odpowiednio częstego wywoływania funkcji aktualizujących obiekty w grze.
W grze ważny jest także silnik graficzny, odpowiadający za wy\'swietlanie obiektów i efektów na ekranie.
Jednym z problemów do rozwiązania jest odpowiednie zsynchronizowanie obu tych elementów z czasem i sobą nawzajem, aby uniknąć sytuacji takich jak:\begin{itemize}
	\item silnik gry zabierający cały czas procesora dla siebie, co powoduje rzadkie odw\'sieżanie zawartości ekranu i niegrywalno\'sć
	\item silnik graficzny zabierający dużo czasu na obliczenia, przez co gra spowalnia i jest dużo mniej grywalna
	\item obliczenia silnika graficznego i silnika gry pokrywające się, przez co niektóre obiekty zostają narysowane przed, a niektóre po aktualizacji stanu, co powoduje błędne wy\'swietanie pozycji niektórych obiektów i może wprowadzić gracza w błąd
\end{itemize}

Innym ważnym problemem jest odpowiednie wykrywanie kolizji w grze, tak aby nie było sytuacji, w której pokrywające się obiekty nie zderzają się ze sobą, albo obiekty będące daleko od siebie mają ze sobą kolizję. Ze względu na to, że to jeden z najważniejszych elemtentów w tego typu grze, potrzebny był algorytm, który będzie mógł szybko i dokładnie to sprawdzić. Zostało to rozwiązane przez podzielenie wykrywania kolizji na fazy oraz zastosowanie odpowiednich algorytmów wykrywania kolizji opisanych w rozdziale \ref{chapter_collission_detection}.

\newpage
\begin{wrapfigure}{l}{0.5\textwidth}
	\centering
	\noindent\includegraphics[width=0.5\textwidth]{player_uml}
	\caption{diagram przedstawiający interakcję gracza z grą}
\end{wrapfigure}
Interakcja gracza z grą jest bardzo prosta.\\
Gracz używa klawiatury do sterowania grą.\\
Jako odpowiedź otrzymuje obraz i dźwięk informujące o tym co aktualnie dzieje się w grze.\\
Do każdego z tych zadań istnieje odpowiednia klasa zajmująca się nim.

Klasy odpowiedzialne za sterowanie oraz logikę gry zostały opisane w rozdziale \ref{chapter_logic}.

Sposób wy\'swietlania na ekranie aktualnej sytuacji, na co składa się narysowanie wszystkich obiektów w grze oraz dodanie efektów graficznych, został opisany w rozdziale \ref{chapter_graphics}.

Dźwięk został opisany w rozdziale \ref{chapter_music}.

Jednym z mniejszych problemów jest dopasowanie do siebie wszystkich elementów gry, czyli przygotowanie gry tak, by podobała się odbiorcy. Przeciwnicy powinni być odpowiednio dopasowani do poziomu zaawansowania gracza, a także nie powinno być zbyt prosto lub zbyt łatwo.

\cleardoublepage