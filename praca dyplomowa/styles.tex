% Domyślny styl dla pracy
\fancypagestyle{custom}{
	\fancyhf{}									% wyczyść stopki i nagłówki
	\fancyhead[RO]{								% Prawy, nieparzysty nagłówek
		\hrulefill \hspace{16pt} \large Rozdział \thechapter
		\put(-472.1, 12.1){%
			\makebox(0,0)[l]{%
				\includegraphics[width=0.05\textwidth]{pwr-logo}
			}
		}
		\put(-443,5.5){%
			\makebox(0,0)[l]{%
				\small Politechnika Wrocławska
			}
		}
	}
	\fancyhead[LE]{								% Lewy, parzysty nagłówek
		\large Rozdział \thechapter \hspace{16pt} \hrulefill 
		\put(-22, 12.1){%
			\makebox(0,0)[l]{%
				\includegraphics[width=0.05\textwidth]{wppt-logo}
			}
		}
		\put(-210,5.5){%
			\makebox(0,0)[l]{%
				\small Wydział Podstawowych Problemów Techniki
			}
		}
	}
	\fancyfoot[LE,RO]{							% Stopki
		\thepage
	}
	\renewcommand{\headrulewidth}{0pt}			% Grubość linii w nagłówku
	\renewcommand{\footrulewidth}{0.2pt}		% Grubość linii w stopce
}


\fancypagestyle{bibliographyStyle}{
	\fancyhf{}									% wyczyść stopki i nagłówki
	\fancyhead[RO]{								% Prawy, nieparzysty nagłówek
		\hrulefill \hspace{16pt} \large Dodatek \thechapter
		\put(-472.1, 12.1){%
			\makebox(0,0)[l]{%
				\includegraphics[width=0.05\textwidth]{pwr-logo}
			}
		}
		\put(-443,5.5){%
			\makebox(0,0)[l]{%
				\small Politechnika Wrocławska
			}
		}
	}
	\fancyhead[LE]{								% Lewy, parzysty nagłówek
		\large Bibliografia \hspace{16pt} \hrulefill 
		\put(-22, 12.1){%
			\makebox(0,0)[l]{%
				\includegraphics[width=0.05\textwidth]{wppt-logo}
			}
		}
		\put(-210,5.5){%
			\makebox(0,0)[l]{%
				\small Wydział Podstawowych Problemów Techniki
			}
		}
	}
	\fancyfoot[LE,RO]{							% Stopki
		\thepage
	}
	\renewcommand{\headrulewidth}{0pt}			% Grubość linii w nagłówku
	\renewcommand{\footrulewidth}{0.2pt}		% Grubość linii w stopce
}

\fancypagestyle{chapterBeginStyle}{
	\fancyhf{}%
	\fancyfoot[LE,RO]{
		\thepage
	}
	\renewcommand{\headrulewidth}{0pt}
	\renewcommand{\footrulewidth}{0.2pt}
}

% Styl dla pozostałych stron spisu treści
\fancypagestyle{tableOfContentStyle}{
	\fancyhf{}%
	\fancyfoot[LE,RO]{
		\thepage
	}
	\renewcommand{\headrulewidth}{0pt}
	\renewcommand{\footrulewidth}{0.2pt}
}

\usepackage{enumitem}

% Nowa lista numerowana z trzema poziomami
\newlist{myitemize}{itemize}{3}

% Definicja wyglądu znacznika pierwszego poziomu
\setlist[myitemize,1]{
	label		=	\textbullet,
	leftmargin	=	4mm}

% Definicja wyglądu znacznika drugiego poziomu
\setlist[myitemize,2]{
	label		=	$\diamond$,
	leftmargin	=	8mm}

% Definicja wyglądu znacznika trzeciego poziomu
\setlist[myitemize,3]{
	label		=	$\diamond$,
	leftmargin	=	12mm
}

\usepackage{graphicx}
\usepackage{footnote}
\usepackage{calc}

\usepackage{hyperref}
\hypersetup{
	colorlinks	=	true,
	linkcolor		=	blue,
	citecolor		=	blue,
	urlcolor		=	blue
}
\urlstyle{same}